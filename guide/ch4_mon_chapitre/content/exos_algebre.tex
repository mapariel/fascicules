\groupexos{Un peu d'algèbre}


	
{\it 
Pour les exercices \ref{cpl1}, \ref{cpl2}, \ref{dv1}, \ref{dv2}, \ref{fact1} et \ref{fact2}, il faut savoir répondre sans la calculatrice. Celle-ci pourra servir à vérifier les réponses.
}



\begin{exo}[type=solution]
Développer en utilisant la simple ou la double distributivité.

\begin{enumerate}[label=\bf{\alph*)\,}]
\item $(2x+3)(x-5)$
\item $3(x-8)-(2x-2)(x+5)$
\item $(x-1)(x^2-2x+3)$
\end{enumerate}


\begin{sol}
On peut facilement vérifier les réponses à l'aide de la calculatrice (menu>algèbre>développer).
\begin{enumerate}[label=\bf{\alph*)\,}]
\item $(2x+3)(x-5)=2x^2-10x+3x-15=2x^2-7x-15$
\item $3(x-8)-(2x-2)(x+5)=3x-24-2x^2-10x+2x+10=-2x^2-5x-14$
\item $(x-1)(x^2-2x+3)=x^3-2x^2-x^2+3x+2x-3=x^3-3x^2+5x-3$
\end{enumerate}

\end{sol}


\end{exo}



\begin{exo}
Développer en utilisant la simple ou la double distributivité.

\begin{enumerate}[label=\bf{\alph*)\,}]
\item $(5-2x)(7x+4)$
\item $8(x^2-3)+(2-x)(2x+7)$
\item $(x-5)(2x^2+3x-8)$
\end{enumerate}

\begin{sol*}
On rappelle la règle de la double distributivité : $(a+b)\cdot(c+d)=a\cdot c+a \cdot d+ b \cdot c+ b \cdot d$.

On peut vérifier les solutions à la calculatrice \texttt{Menu > algèbre > développer}
\end{sol*}

\end{exo}





\begin{exo}[type=solution] \label{dv2}
	Développer en utilisant les identités remarquables
	
\begin{multicols}{2}
	\begin{enumerate}[label=\bf{\alph*)\,}]		
\item		$(x-4)^2$		
\item		$(x+5)(x-5)$		
\item		$(2x+3)^2$		
\item		$(6x+1)^2$		
\item		$(3x-2)(3x+2)$		
\item		$(4x-7)^2$
\end{enumerate}		
\end{multicols}

\begin{sol}
\begin{enumerate}[label=\bf{\alph*)\,}]		
\item		$(x-4)^2 = x^2-8x+16$		
\item		$(x+5)(x-5)=x^2-25$		
\item		$(2x+3)^2=4x^2+12x+9$		
\item		$(6x+1)^2=36x^2+12x+1$		
\item		$(3x-2)(3x+2)=9x^2-4$		
\item		$(4x-7)^2=16x^2-56x+49$
\end{enumerate}		
\end{sol}	


\end{exo}





\begin{exo} \label{dv1}
Développer en utilisant les identités remarquables

\begin{multicols}{2}
\begin{enumerate}[label=\bf{\alph*)\,}]
\item $(x+3)^2$

\item $(2x+1)^2$

\item $(x-2)(x+2)$

\item $(3x-5)^2$

\item  $(7x-2)^2$

\item  $(2x-1)(2x+1)$

\end{enumerate}

\end{multicols}

\begin{sol*}
On rappelle les trois identités remarquables :
$(a+b)^2 = a^2+2ab+b^2$, $(a-b)^2 = a^2-2ab+b^2$ et $(a+b)(a-b) = a^2-b^2$.

On peut aussi vérifier les résultats des développements à la calculatrice. 
\end{sol*}

\end{exo}	



\begin{exo}[type=solution]  Recopier et compléter les expressions. \label{cpl2}
	
	\begin{enumerate}[label=\bf{\alph*)\,}]		
		\item		$x^2-\dashuline{\hspace{0.8cm}}+\dashuline{\hspace{0.8cm}}=(\dashuline{\hspace{0.8cm}}-4)^2$	
		\item		$\dashuline{\hspace{0.8cm}}+4x+\dashuline{\hspace{0.8cm}}=(2x+\dashuline{\hspace{0.8cm}})^2$
		\item		$\dashuline{\hspace{0.8cm}}-8x+\dashuline{\hspace{0.8cm}}=(\dashuline{\hspace{0.8cm}}-2)^2$
		\item		$\dashuline{\hspace{0.8cm}}+18x+\dashuline{\hspace{0.8cm}}=(x+\dashuline{\hspace{0.8cm}})^2$
	\item $   (\dashuline{\hspace{.8cm}}+\dashuline{\hspace{.8cm}})(\dashuline{\hspace{.8cm}}-8)= 100x^2-\dashuline{\hspace{1cm}}$ 	
	\end{enumerate}
	
	
\begin{sol}
	\begin{enumerate}[label=\bf{\alph*)\,}]		
		\item		$x^2-\textcolor{red}{8x}+\textcolor{red}{16}=(\textcolor{red}{x}-4)^2$
		\item		$\textcolor{red}{4x^2}+4x+\textcolor{red}{1}=(2x+\textcolor{red}{1})^2$
		\item		$\textcolor{red}{4x^2}-8x+\textcolor{red}{4}=(\textcolor{red}{2x^2}-2)^2$
		\item		$\textcolor{red}{x^2}+18x+\textcolor{red}{81}=(x+\textcolor{red}{9})^2$
		\item $   (\textcolor{red}{10x}+\textcolor{red}{8})(\textcolor{red}{10x}-8)= 100x^2-\textcolor{red}{64}$ 	
	\end{enumerate}	
\end{sol}		
\end{exo}	





\begin{exo} Recopier et compléter les égalités.
	\label{cpl1}
	
\begin{enumerate}[label=\bf{\alph*)\,}]		
		
\item 		$(x+\dashuline{\hspace{0.8cm}})^2 = \dashuline{\hspace{0.8cm}}+\dashuline{\hspace{0.8cm}}+4$
		
\item		$(x-\dashuline{\hspace{0.8cm}})^2=\dashuline{\hspace{0.8cm}}-\dashuline{\hspace{0.8cm}}+25$
		
\item		$(\dashuline{\hspace{0.8cm}}+\dashuline{\hspace{0.8cm}})^2 = 9x^2+\dashuline{\hspace{0.8cm}}+25$
		
\item		$(x+\dashuline{\hspace{0.8cm}})^2= \dashuline{\hspace{0.8cm}}+14x+\dashuline{\hspace{0.8cm}}$
		
\item $   (\dashuline{\hspace{.8cm}}+5)(\dashuline{\hspace{.8cm}}-\dashuline{\hspace{.8cm}})= 16x^2-\dashuline{\hspace{1cm}}$ 
\end{enumerate}


\begin{sol*}

\begin{enumerate}[label=\bf{\alph*)\,}]		
		
\item 		$(x+\dashuline{\textcolor{red}{2}})^2 = \dashuline{\textcolor{red}{x^2}}+\dashuline{\textcolor{red}{4x}}+4$
		
\item		$(x-\dashuline{\textcolor{red}{5}})^2=\dashuline{\textcolor{red}{x^2}}-\dashuline{\textcolor{red}{10x}}+25$
		
\item		$(\dashuline{\textcolor{red}{3x}}+\dashuline{\textcolor{red}{5}})^2 = 9x^2+\dashuline{\textcolor{red}{30x}}+25$
		
\item		$(x+\dashuline{\textcolor{red}{7}})^2= \dashuline{\textcolor{red}{x^2}}+14x+\dashuline{\textcolor{red}{49}}$
		
\item $   (\dashuline{\textcolor{red}{4x}}+5)(\dashuline{\textcolor{red}{4x}}-\dashuline{\textcolor{red}{5}})= 16x^2-\dashuline{\textcolor{red}{25}}$ 
\end{enumerate}


\end{sol*}

\end{exo}	







\begin{exo}[type=solution] Factoriser les expressions suivantes: 
\label{fact1}

\begin{multicols}{2}
\begin{enumerate}[label=\bf{\alph*)\,}]		
\item  $x^2+10x+25$
\item  $4x^2-20x+25$
\item  $x^2+6x+9$
\item $(x+1)^2-4$
\item  $x^2+24x+144$
\item $x^2-100$
\end{enumerate}
\end{multicols}

\begin{sol}
On peut vérifier facilement les réponses à l'aide de la calculatrice : \texttt{Menu>algèbre>factoriser}
\begin{enumerate}[label=\bf{\alph*)\,}]		
\item  $x^2+10x+25=(x+5)^2$
\item  $4x^2-20x+25 = (2x-5)^2$
\item  $x^2+6x+9 = (x+3)^2$
\item $(x+1)^2-4 =(x+1+2)(x+1-2)=(x+3)(x-1)$
\item  $x^2+24x+144 = (x+12)^2$
\item $x^2-100=(x+10)(x-10)$
\end{enumerate}
\end{sol}
\end{exo}	





\begin{exo}
Factoriser les expressions suivantes \label{fact2}

\begin{multicols}{2}
	\begin{enumerate}[label=\bf{\alph*)\,}]	
\item  $x^2-2x+1$
\item  $4x^2+12x+9$
\item $x^2-64$
\item $36x^2-12x+1$
\item  $9x^2-18x+9$
\item $(x-5)^2-36$
\end{enumerate}
\end{multicols}

\begin{sol*}
Ici encore, on peut vérifier les résultats à la calculatrice : \texttt{menu > algèbre > factoriser}.
\end{sol*}

\end{exo}	