{
	\relsize{+1}
	


\begin{activity}{TP Geogebra. Les fonctions du second degré.}
\label{second_degre_TP}




\begin{objective}
Explorer les fonctions du second degré. Travailler sur la forme développée et la forme canonique.
\end{objective}


\subsection*{fonctions $x \mapsto a \cdot x^2$}  

\begin{itemize}
\item Tracer sur le même graphique  les courbes d'équations  suivantes : $y=x^2$, $y=2x^2$, $y=-x^2$ et $y=-3x^2$. Faire apparaître les équations des courbes sur le graphique.
\end{itemize}

\begin{enumerate}
\item Faire une capture  d'écran des quatre courbes.
\end{enumerate}


\begin{itemize}

\item Effacer les 4 courbes précédentes. Créer un curseur compris entre $-10$ et $10$, puis la courbe d'équation $y=a \cdot   x^2$. Puis déplacer le curseur.

\end{itemize}

\begin{enumerate}[resume]
\item A quelle condition la courbe d'équation $y=a \cdot x^2$ a-t-elle une forme <<creuse>> ou bien une forme <<bosse>> ?
\end{enumerate}





\medskip

\subsection*{Évolution d'une parabole}

\begin{itemize}
\item Ouvrir une nouvelle fenêtre (inutile de sauvegarder).

\item Créer 3 curseurs $a$, $r$ et $s$.

\item Créer la fonction $p(x)=a(x-s)^2+r$

\item Déplacer les trois curseurs.

\end{itemize}

\begin{enumerate}
[resume]

\item Faire une capture d'écran du tracé de la fonction $p$ avec les trois curseurs.

\item Quel effet observe-t-on sur la courbe lorsqu'on déplace le curseur $a$ ? le curseur $s$ ? le curseur $r$ ? 



\item Recopier et compléter le tableau suivant :


\begin{tabular}{|c|c|c|} \hline 
& forme <<creuse>> ou <<bosse>> & coordonnées du sommet \\ \hline 
$a=1$, $s=3$ et  $r=5$ & & \\ \hline 
$a=-1$, $s=-1$ et  $r=10$ & & \\ \hline 
$a=3$, $s=-5$ et  $r=-6$ & & \\ \hline 
$a=-2$, $s=10$ et  $r=-8$ & & \\ \hline 
\end{tabular}

\end{enumerate}


\bigskip

\subsection*{Forme canonique}

\begin{itemize}

\item  Tracer la parabole qui représente la fonction  $f(x)=x^2-6x+5$.

\item Déplacer les 3 curseurs pour que les courbes de $f$ et de $p$ se superposent. 

\end{itemize}



\begin{enumerate}[resume]
\item Faire une capture d'écran.

\item Donner les valeurs des 3 curseurs puis l'expression de $p(x)$

\item Montrer en détaillant le calcul que  $f(x)=p(x)$. 

\item Par lecture graphique, indiquer les coordonnées du sommet de la parabole.

\item Résoudre graphiquement l'équation $x^2-6x+5=0$.

\item Essayer de résoudre cette équation par le calcul.

\end{enumerate}

\end{activity}

}