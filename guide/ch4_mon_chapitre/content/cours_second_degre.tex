\tagged{method}{\begin{xcomment}{method,exo}}

\section{Généralités.}
\label{second_degre_generalites}

\subsection*{Définition}

\begin{frame}[t]

\onslide<1->{
\begin{definition}{}
 Une fonction du second degré est de la forme $f(x)=ax^2+bx+c$. 
\onslide<2->{ Les nombres $a$, $b$ et $c$ sont  appelés les \underline{coefficients}. }
\onslide<3->{  Le coefficient $a$ doit être différent de zéro. }
\end{definition}
}

\onslide<4->{
Exemple : 

\begin{itemize}
	\item<5-> $2x^2-5x+10$ est une fonction du second degré. \onslide<6->{Ses coefficients sont 2,$-5$ et 10. }
	
	\item<7-> $5x-3$ n'est pas une fonction du second degré. \onslide<8->{C'est une fonction affine.}
	
	\item<9-> $x^3-5x+2$ non plus. \onslide<10->{C'est un polynôme de degré 3.}
\end{itemize}
}

\onslide<11->{\notez}

\end{frame}


\subsection*{Forme convexe et concave}

\begin{frame} 

\begin{property}{}
\onslide<1->{
La courbe représentative d'une fonction du second degré est une parabole \underline{convexe} si $a$ est positif} \onslide<3->{ et \underline{concave} si $a$ est négatif.}	

\onslide<5->{La parabole a un axe de symétrie qui passe par le sommet.}
\end{property}


\begin{center}
\begin{tabular}{|c|c|} \hline 
	\onslide<2->{forme convexe} & \onslide<5->{forme concave} \\
	
	\begin{tikzpicture}[scale=1,general]
	\window{-3}{3}{-2}{5}
	\begin{windowsratio}
	\onslide<2->{
	\axeH;
	\clip (\Xmin,\Ymin) rectangle (\Xmax,\Ymax);
	\def \f{0.5*(\x)^2-1};
	\draw[samples=100,domain=-4:4,courbe]
	plot(\x,{\f});
	}
	\onslide<6->{
	\draw[color=green!60!black] plot[mark=*,mark options={scale=2}] coordinates {(0,-1)} node[below right]{$S$};
	\draw[color=green!60!black,dashed] (0,-2)--(0,5); 
	\draw[color=green!60!black,dashed] (0,-1)--(0,0) node[above left]{$-\frac{b}{2a}$};  
	}
	\end{windowsratio}
	\end{tikzpicture} &

	\begin{tikzpicture}[scale=1,general]
	\onslide<4->{
	\window{-3}{3}{-2}{5}
	\begin{windowsratio}
	\axeH;
	\clip (\Xmin,\Ymin) rectangle (\Xmax,\Ymax);
	\def \f{-0.5*(\x)^2+3};
	\draw[samples=100,domain=-4:4,courbe]
	plot(\x,{\f});
	}
	\onslide<7->{
	\draw[color=green!60!black] plot[mark=*,mark options={scale=2}] coordinates {(0,3)} node[above right]{$S$};
	\draw[color=green!60!black,dashed] (0,-2)--(0,5); 
	\draw[color=green!60!black,dashed] (0,3)--(0,0) node[below]{$-\frac{b}{2a}$}; }           
	\end{windowsratio}
	\end{tikzpicture} 	
 \\ \hline
\end{tabular}	
\end{center}

\onslide<8->{\notez}

\end{frame}





\section{Tout est dans le style \ldots}
\label{second_degre_style}

\begin{frame}

	Une fonction du second degré peut s'écrire de trois manières différentes :
	
	 
	\onslide<2->
	\hfil
	\begin{tikzpicture}
	\onslide<4->{
	\node[ellipse,fill=green!40!white] (f1) at (0,0) 
	{\begin{tabular}{cc}forme développée \\ $ax^2+bx+c$ \end{tabular}};
	}
	\onslide<7->{
	\node[ellipse,fill=blue!20!white] (f2) at (6,0) 	{\begin{tabular}{cc}forme factorisée \\ $a(x-x_1)(x-x_2)$ \end{tabular}};
	}
	
	\onslide<2->{
	\node[ellipse,fill=red!40!white] (f3) at (3,-3) 	{\begin{tabular}{cc}forme canonique \\ $a(x-r)^2+s$ \end{tabular}};
	}
	
	\onslide<11->{
	\draw[-latex] (f1) to[bend right] node[above] {fact.} (f2);
	\draw[-latex] (f2) to[bend right] node[below] {dvp.} (f1);
    }
	\onslide<9->{\draw[-latex] (f1.south west) to[bend right] node{compléter le carré} (f3.south west);}
	\onslide<3->{ \draw[-latex] (f3.north west) to[bend right] node{développer} (f1.south); }
	\onslide<10->{
	\draw[-latex] (f2.south) to[bend right] node {comp. le carré} (f3.north east); }
	\onslide<6->{
	\draw[-latex] (f3.south east) to[bend right] node{factoriser} (f2.south east);
	}
	\end{tikzpicture}
	\hfil

Exemple : $\onslide<5->{2x^2-20x+32=} \onslide<2->{2(x-5)^2-18} \onslide<8->{=2(x-8)(x-2)}$.

\onslide<12->{	\notez}
	
	
\end{frame}


\section{L'équation du second degré}
\label{second_degre_formule_resolution}


\begin{frame}

Pour résoudre l'équation $ax^2+bx+c=0$ on commence par calculer le \underline{discriminant} $$\Delta=b^2-4ac \,.$$



\begin{theorem}{}
\begin{itemize}
\item<2-> Si $\Delta>0$ alors l'équation admet deux solutions:

 $x_1=\frac{-b-\sqrt{\Delta}}{2a}$ et $x_2=\frac{-b+\sqrt{\Delta}}{2a}$.

\item<4-> Si $\Delta=0$ alors l'équation admet une seule solution
 $x_0=\frac{-b}{2a}$.

\item<5-> Si $\Delta<0$ alors l'équation n'admet aucune solution.

\end{itemize}
\end{theorem}
	
\mode<beamer>{
\onslide<3>{\color{green} C'est le cas qu'on rencontre le plus souvent.	
}	}

\onslide<6> \notez
	
	
\end{frame}	





\begin{frame}


\begin{method}[methode_delta]{Trouver les zéros de $ax^2+bx+c$}
Cela revient à résoudre l'équation $ax^2+bx+c=0$. On commence par calculer le discriminant $\Delta$, puis on utilise les formules.
\end{method}

\begin{exo}[type=method]
Résoudre l'équation $x^2-2x-15=0$.

\begin{sol}

\onslide<2->{	
{\color<beamer>{green} Ici $a=1$, $b=-2$ et $c=-15$,}}
\onslide<3->{ 
{\color<beamer>{green}donc} $$\Delta=(-2)^2-4 \cdot 1 \cdot (-15) = 4+60 = 64 \,.$$ }

\vspace{-5mm}

\onslide<4->{
$\Delta$ est positif donc l'équation admet deux solutions:}

\begin{center}
$
\begin{array}[t]{rcl}
\onslide<5->{ x_1 & = & \frac{2-\sqrt{64}}{2\cdot 1}  }\\
\onslide<6->{& = & \frac{2-8}{2}} \\
\onslide<7->{ & = & \frac{-6}{2} =-3 }
\end{array}
$ \onslide<5->{et}
$
\begin{array}[t]{rcl}
\onslide<5->{x_2 & = & \frac{2+\sqrt{64}}{2\cdot 1}} \\
\onslide<6->{ & = & \frac{2+8}{2}  }\\
\onslide<7->{ & = & \frac{10}{2} =5 }
\end{array}
$
\end{center}


\onslide<8->{
\begin{remark}On peut faire une vérification:\end{remark} }


\onslide<9->{ Pour $x=-3$, \qquad $x^2-2x-15} \onslide<10->{ = 9 + 6 -15} \onslide<11->{= 0.}$ 

\onslide<12->{Pour $x=5$,  \qquad  $x^2-2x-15 } \onslide<13->{ = 25 -10 -15 } \onslide<14->{= 0.}$


\onslide<15->{\notez}
\end{sol}

\end{exo}



\end{frame}












\section{Formule du sommet}
\label{second_degre_formule_sommet}


\begin{frame}
	
	Lorsque la fonction est sous la forme canonique $a(x-r)^2+s$, le sommet est $S(r;s)$.  
	
	\pause Sinon :
	
	\pause 
	
	\begin{formula}{}
		Le sommet de la parabole, d'équation $y=ax^2+bx+c$, a pour abscisse $x_S = -\frac{b}{2a}$.
	\end{formula}
	
	\pause 
	
	\notez
	
\end{frame}


\begin{frame}
	
	\begin{method}[methode_sommet]{Coordonnées du sommet}
	Soit on sait mettre le trinôme sous la forme canonique, soit on applique la formule.
	\end{method}
	
	\begin{exo}[type=method]	
		Calculer les coordonnées du sommet de la parabole d'équation $y=\alert<5>{2}x^2\alert<4>{-8}x+5$.
		
		\begin{sol}
		
		\onslide<2->
		{$x_S}  \onslide<3->{= -  \frac{\onslide<4->{\alert<4>{-8}}}{\onslide<5->{2 \cdot \alert<5>{2} }}}  \onslide<6->{= 2 .} $
		
		\onslide<7->{Pour $x=2$,}  $\onslide<8->{y = 2\cdot 2^2 -8 \cdot 2 +5 } \onslide<9->{ = 
			8-16+5 } \onslide<10->{= -3.}$
		
		\onslide<11->{Donc le sommet de la parabole est $S(2;-3)$.}
		
		\onslide<12>{\notez}
		\end{sol}
		
		\clearpage
		
	\end{exo}	
\end{frame}

\tagged{method}{\end{xcomment}}











