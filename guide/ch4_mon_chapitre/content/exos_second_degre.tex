
\groupexos{Étude graphique}



\begin{exo}[type=solution][Sans calculatrice.]
Soit $P(x)=x^2-2x-3$.

\begin{enumerate}
\item Calculer $P(0)$ et $P(1)$.
\item Recopier et compléter de valeurs suivant, sans détailler les calculs.

{
\setlength{\arraycolsep}{2mm}
$$
\begin{array}{|c|c|c|c|c|c|c|c|} \hline 
x & -2 & -1 & 0 & 1& 2 & 3 & 4  \\ \hline
P(x) & & & & & & &  \\ \hline
\end{array}
$$
}
\item Tracer la courbe représentant la fonction  (on prendra 1 gros carreau pour une unité en abscisses et en ordonnées.)

\smallskip
{\it On répondra aux questions suivantes sans justification, uniquement en observant le graphique.}
\smallskip

\item Déterminer les coordonnées du sommet.
\item Tracer l'axe de symétrie et indiquer son équation sur le dessin. 
\item Déterminer les coordonnées de l'intersection de la courbe avec l'axe des ordonnées.
\item Déterminer les zéros de la fonction?

\end{enumerate}

\begin{sol}
\begin{enumerate}
\item $P(0)=0-0-3=-3$ et $P(1)=1-2-3=-4$.
\item Tableau de valeurs : 
$$
\begin{array}{|c|c|c|c|c|c|c|c|} \hline 
x & -2 & -1 & 0 & 1& 2 & 3 & 4  \\ \hline
P(x) & 5  & 0 & -3 &  -4 & -3 & 0 & 5  \\ \hline
\end{array}
$$
\item Graphique 

\begin{tikzpicture}[scale=1,general]
    \window{-3}{5}{-5}{6} 
    \begin{windowsratio}
    \draw[xstep=1,ystep=1,grid] (\Xmin,\Ymin) grid (\Xmax,\Ymax);
    \axeH;\axeV;\tickY[1];
    \node[below left] at (0,0) {0};
    \node[below left] at (1,0) {1};
    \clip (\Xmin,\Ymin) rectangle (\Xmax,\Ymax);
    \def \f{(\x)^2-2*\x-3};
    \draw[samples=100,domain=-3:6,courbe]
       plot(\x,{\f});
     \draw[courbe,color=green!50!black] (1,-5)--(1,6)node[below right]{$x=1$};       
    \end{windowsratio}
\end{tikzpicture}
\item Le sommet a pour coordonnées $(1;-4)$.
\item L'équation de l'axe de symétrie est $x=1$.
\item L'intersection de la courbe avec l'axe des $y$ est $(0;-3)$.
\item Les zéros de $P$ sont $-1$ et $3$.
\end{enumerate}
\end{sol}

\end{exo}






\begin{exo}[type=solution,title=Avec la calculatrice.]

Soit la fonction $f$ définie par $f(x)=-2x^2+6x+8$. 

{\it Dans cet exercice, on répondra à l'aide de la calculatrice, sans détailler les calculs.}


\begin{enumerate}

\item Tracer à la calculatrice la courbe de $f$ pour $x$ compris entre $-2$ et $5$. 
\item Établir le tableau de valeurs de $f$ pour $x$ compris entre $-2$ et 5, avec un pas de 1.
\item Déterminer  les coordonnées du sommet $S$ de la parabole.
\item Déterminer l'équation de l'axe de symétrie.
\item Déterminer les coordonnées du point $A$ intersection de la courbe avec l'axe des $y$.
\item Déterminer les zéros de la fonction?
\item Tracer dans un repère la courbe représentant la fonction $f$. On placera les points correspondant au tableau de valeurs, le sommet $S$, le point $A$, l'axe de symétrie (avec son équation) et les deux zéros.

On pourra prendre 1 carreau pour une unité en abscisse et 1 carreau pour 5 unités en ordonnées.

\end{enumerate}

\begin{sol}
\begin{enumerate}
\item On peut prendre par exemple $x$ compris entre $-2$ et $5$ et $y$ compris entre $-7$ et $20$ (\texttt{Menu>Fenêtre Zoom>Réglage de la fenêtre}).
\item 
$
\begin{array}{|c|c|c|c|c|c|c|c|c|}\hline 
x & -2 &-1 & 0 & 1 & 2 & 3 & 4 & 5 \\ \hline 
f(x) & -12 & 0 & 8  & 12 & 12 & 8 & 0 & -12 \\  \hline 
\end{array}
$

\item Le sommet $S$ a pour coordonnées $(1,5;12,5)$.
\item L'axe de symétrie a pour équation $x=1,5$.
\item L'intersection avec l'axe des $y$ est le point $(0;8)$.
\item Les zéros sont $-1$ et $4$ (il suffit de regarder dans le tableau de valeurs).
\item Graphique de $f$:

\begin{tikzpicture}[scale=1,general]
\window{-2}{5}{-10}{20} 
\begin{windowsratio}
\draw[xstep=1,ystep=5,grid] (\Xmin,\Ymin) grid (\Xmax,\Ymax);
\axeH;\axeV;\tickX;\tickY[20];
\node[below left] at (0,0) {0 };
\clip (\Xmin,\Ymin) rectangle (\Xmax,\Ymax);
\def \f{-2*(\x)^2+6*\x+8};
\draw[samples=100,domain=-2:5,courbe]
plot(\x,{\f});    

\draw plot[mark=*,mark options={yscale=8}] coordinates{(1.5,12.5)} node[above right]{$S$}; 

\draw plot[mark=*,mark options={yscale=8}] coordinates{(0,8)} node[above left]{$A$}; 
\draw plot[mark=*,mark options={yscale=8}] coordinates{(-1,0)};
\draw plot[mark=*,mark options={yscale=8}] coordinates{(4,0)};

\draw[color=red] (1.5,-10)--(1.5,20) node[below right]{$x=1.5$};

\end{windowsratio}
\end{tikzpicture}

\end{enumerate}	
\end{sol}	

\end{exo}





\begin{exo}
Soit $g(x)=4x^2-160x+1200$. Tracer à la calculatrice la courbe de $f$ pour $x$ compris entre $0$ et $40$. 

{\it Dans cet exercice, on répondra à l'aide de la calculatrice, sans détailler les calculs.}


\begin{enumerate}
\item Déterminer  les coordonnées du sommet.

\item Déterminer l'équation de l'axe de symétrie.

\item Déterminer les coordonnées de l'intersection de la courbe avec l'axe des ordonnées.

\item Déterminer les zéros de la fonction?

\item Tracer la courbe de la fonction pour $x$ compris entre 0 et 40. On prendra 2 carreaux pour 10 unités en abscisse et 1 carreau pour 200 unités en ordonnée. 

On placera aussi précisément que possible les points déterminés plus haut et l'axe de symétrie.
\end{enumerate}


%
%
%\begin{image}
%\begin{tikzpicture}[scale=1,general]
%    \window{0}{40}{-60}{140} 
%    \begin{windowsratio}
%    \draw[xstep=10,ystep=20,grid] (\Xmin,\Ymin) grid (\Xmax,\Ymax);
%    \axeH;\axeV;\tickX[10];
%    \draw (0,40) node {\small $+$} node[left]{400};
%    \node[below left] at (0,0) {0 };
%    \clip (\Xmin,\Ymin) rectangle (\Xmax,\Ymax);   
%    \end{windowsratio}
%\end{tikzpicture}
%\end{image}

\end{exo}





{\it Dans les exercices \ref{roxane} et \ref{rugby}, on répondra à l'aide de la calculatrice, on essaiera de donner des explications mathématiques mais les détails des  calculs ne sont pas nécessaires.}	


\begin{exo}
	Roxane lance, du premier étage de son lycée, une balle de tennis à son amie Leila, qui la réceptionne sur la pelouse. La trajectoire de la balle est la parabole d'équation: \[y=-0,084x^2+0,839x+5 \,,\]

\label{roxane}	
	
	L'origine du repère est située au niveau du sol, sous les pieds de Roxane. Elle lance la balle dans la direction des $x$ positifs.
	

	\begin{enumerate}
		\item Quelle est la hauteur maximale atteinte par la balle ?
		\item Quelle est la distance entre le pied du mur et l'endroit où la balle touche le sol ?
	\end{enumerate}
\end{exo}





\begin{exo}
	Au moment du coup de pied, un ballon de rugby se trouve au sol, en O, face aux poteaux, à une distance de 50m. On admet que la trajectoire du ballon est la parabole d'équation $y=-0,02x^2+1,19x$.

\label{rugby}	

\begin{tikzpicture}[scale=1,general]
\window{-1}{6}{-1}{3}
\begin{windowsratio}
%\draw[xstep=1,ystep=10,grid] (\Xmin,\Ymin) grid (\Xmax,\Ymax);
\axeH;\axeV;\tickX;\tickY;
\node[below left] at (0,0) {0 };
\clip (\Xmin,\Ymin) rectangle (\Xmax,\Ymax);
	\begin{scope}[color=orange,line width=1pt]
	\draw (4.5,.5)--(4.5,2.5);
	\draw (4.5,1.5)--(5.2,0.5);
	\draw (5.2,-.5)--(5.2,1.5);
	\end{scope}  
	\node[below left] at(0,0) {$O$};
	\draw[dashed] (3.8,1.5)--(5.9,-1.5);
	\draw[dashed] (-1.05,1.5)--(1.05,-1.5);
	\draw[<->] (.8,-1)--node[above]{50 m} (5.5,-1) ;
\end{windowsratio}
\end{tikzpicture}


%	
%	\begin{tikzpicture}[scale=1]
%	\coordinate  (so) at (0,0); % coin inférieur gauche
%	\coordinate  (ne) at (6,3); % coin supérieur droit
%	\draw[axe] ( so|-0,0 )-- ( ne|-0,0 ) node[right]{$x$};
%	\draw[axe] ( so-|0,0 )-- ( ne-|0,0 ) node[above]{$y$};
%
%	\end{tikzpicture}
	
	
	
	
	\begin{enumerate}
		
		\item La pénalité est réussie si le ballon passe au-dessus de la barre située à 3m de haut. Le buteur a-t-il réussi la pénalité ?
		
		\item Quelle est la hauteur maximale atteinte par le ballon ?
		
		\item A combien de mètres derrière la ligne de but le ballon touche-t-il le sol pour la première fois ?
		
	\end{enumerate}	

\begin{sol*}
	\begin{enumerate}
	\item Pour $x=50$, $y=-0,02x^2+1,19x = -50+59,5 = 9,5$. Donc la balle passe largement au-dessus de la barre située à une hauteur 3m. La pénalité est donc réussie.
	\item D'après la calculatrice, le sommet de la courbe est $(29,7;17,7)$. Donc la hauteur maximale de la balle est de 17,7 m.
	\item $-0,02x^2+1,19x = 0$ si $x=0$ ou si $x=59,5$. La balle retombe 9,5 m derrière la ligne de but.
	\end{enumerate}		
\end{sol*}	
	
\end{exo}



\groupexos{Résolution de $ax^2+bx+c=0$}



%\begin{exom}{\refmethode{methode_delta}}
%
%Résoudre l'équation $x^2-2x-15=0$.
%\end{exom}

\begin{exo}[type=solution]
Résoudre les équations suivantes en utilisant la formule du discriminant.

\vspace{-2mm}
\begin{multicols}{2}
\begin{enumerate}[label=\bf{\alph*})\,]

\item $2x^2+10x-100=0$

\item $-x^2+15x-36=0$

\item $x^2+16x+64=0$

\item $5x^2-10x+20=0$
\end{enumerate}
\end{multicols}

\begin{sol}
\begin{enumerate}[label=\bf{\alph*})\,]
\item $\Delta=100+4\cdot2\cdot100=900$. 

Il y a deux solutions $x_1=\frac{-10+\sqrt{900}}{2\cdot 2} = \frac{-10+30}{4} = \frac{20}{4} = 5$ et
$x_2=\frac{-10-\sqrt{900}}{2\cdot 2} = \frac{-10-30}{4} = \frac{-40}{4} = -10$.


\item $\Delta = 15^2-4 \cdot 36 =81$.

Les solutions sont $x_1= \frac{-15+\sqrt{81}}{-2} = \frac{-6}{-2} = 3$ et $x_2 =\frac{-15-\sqrt{81}}{-2}= \frac{-24}{-2}=12$. 


\item $\Delta = 16^2-4\cdot 64 = 0$. Il y a une seule solution $x_0=\frac{-16}{2} = -8$.


\item $\Delta = 10^2-4\cdot 5 \cdot 20 = -300$. Comme le discriminant est négatif, il n'y a pas de solution.

\end{enumerate}	
\end{sol}	

\end{exo}



\begin{exo}
Sans la calculatrice, résoudre les équations suivantes en utilisant la formule du discriminant.

\vspace{-2mm}

\begin{multicols}{2}
\begin{enumerate}[label=\bf{\alph*})\,]
\item $x^2+7x+10=0$
\item  $x^2-2x-3=0$
\item  $-x^2+2x+8=0$
\item  $x^2+4x-5=0$
\item  $-x^2+3x+10=0$
\item  $x^2-4x=0$
\end{enumerate}
\end{multicols}

\begin{sol*}
\begin{enumerate}[label=\bf{\alph*})\,]
	\item $x^2+7x+10=0$
	
	$\Delta = 49-4 \cdot 1 \cdot 10 = 9$. Donc il y a deux solutions \par $x_1=\frac{-7+3}{2}=-2$ et $x_2=\frac{-7-3}{2}=-5$. 
	
	\item  $x^2-2x-3=0$
	
	$\Delta = 4+4 \cdot 1 \cdot 3=4+12=16$. Les deux solutions sont \par $x_1 = \frac{2+4}{2} = \frac{6}{2} =3$ et $x_2 = \frac{2-4}{2} = \frac{-2}{2} =-1$
	
	\item  $-x^2+2x+8=0$
	
	$\Delta = 4+4 \cdot 1 \cdot 8=4+32=36$. Les deux solutions sont \par $x_1 = \frac{-2+6}{-2} = \frac{4}{-2} =-2$ et $x_2 = \frac{-2-6}{-2} = \frac{-8}{-2} =4 $	
	
	
	\item  $x^2+4x-5=0$
	
	$\Delta = 16+4 \cdot 1 \cdot 5=16+20=36$. Les deux solutions sont \par $x_1 = \frac{-4+6}{2} = \frac{2}{2} =1$ et $x_2 = \frac{-4-6}{2} = \frac{-10}{2} =-5 $	
	
	
	\item  $-x^2+3x+10=0$
	
	$\Delta = 9+4 \cdot 1 \cdot 10=9+40=49$. Les deux solutions sont \par $x_1 = \frac{-3+7}{-2} = \frac{4}{-2} =-2$ et $x_2 = \frac{-3-7}{-2} = \frac{-10}{-2} =5 $	
	
	
	\item  $x^2-4x=0$
	
	$\Delta = 16+4 \cdot 1 \cdot 0=9+40=16$. Les deux solutions sont \par $x_1 = \frac{4+4}{2} = \frac{8}{2} =4$ et $x_2 = \frac{4-4}{2} = 0 $	
	
	
\end{enumerate}	
\end{sol*}	


\end{exo}



\begin{exo}[type=solution]
 Résoudre les équations suivantes, en utilisant le discriminant pour \textbf{une seule} d'entre elles.
	
	\begin{multicols}{2}
	\begin{enumerate}[label=\bf{\alph*})\,]
		\item $x^2+6x=0$
		
		\item $2x^2-4x+6 = 0$
		
		\item $x^2-4=0$
		
		\item $3x^2-4x=0$
		
		\item $(2x+1)(x+4)=0$
		
		\item $x^2-4x+4=0$
		
	\end{enumerate}	
	\end{multicols}


\begin{sol}
\begin{enumerate}[label=\bf{\alph*})\,]
\item 
$
\begin{array}[t]{rcl}
x^2+6x &  = & 0 \\ 
x(x+6) & = & 0  \\
 x=0  & \text{ ou }& x+6=0 \\
  & & x=-6 \\
\end{array}
$
\item 
$2x^2-4x+6 = 0$.

$\Delta = 16-4 \cdot 6 \cdot 2 = 16-48 = -32$.  $\Delta$ est négatif donc il n'y a pas de solutions.

\item 
$
\begin{array}[t]{rcl}
	x^2-4 &  = & 0 \\ 
	x^2 & = & 4  \\
	x=-2  & \text{ ou }& x=2 \\
\end{array}
$
\item 
$
\begin{array}[t]{rcl}
3x^2-4x &  = & 0 \\ 
x(3x-4) & = & 0  \\
x=0  & \text{ ou }& 3x-4=0 \\
     & \text{ ou }& x=\frac{4}{3} \\
\end{array}
$
\item 
$
\begin{array}[t]{rcl}
(2x+1)(x+4) &  = & 0 \\ 
2x+1=0 & \text{ ou } & x+4=0  \\
x=-\frac{1}{2}  & \text{ ou }& x=-4 \\
\end{array}
$
\item 
$
\begin{array}[t]{rcl}
x^2-4x+4 &  = & 0 \\ 
(x-2)^2 & = & 0  \\
x-2 & = & 0  \\
x & = & 2 \\
\end{array}
$





\end{enumerate}	 	
\end{sol}	

	
\end{exo}	










\begin{exo}
Résoudre les équations suivantes en utilisant la formule du discriminant. On donnera une valeur exacte puis approchée  à 0,01 près des solutions.

\vspace{-2mm}
\begin{multicols}{2}
\begin{enumerate}[label=\bf{\alph*})\,]

\item $3x^2-6x+2=0$

\item $5x^2-6x+1,8=0$

\item $-4x^2+3x-5=0$

\item $x^2+10x-100=0$

\item $4x^2-10x+7=0$

\item $-2x^2+7x+4=0$
\end{enumerate}
\end{multicols}
\end{exo}


\groupexos{Équations se ramenant au second degré}


\begin{exo}[type=solution]
Sans calculatrice, résoudre les équations suivantes :

\begin{multicols}{2}
\begin{enumerate}[label=\bf{\alph*})\,]
\item $x^2+2x=3$
\item $x^2+40 = -13x$
\item $x^2+5x = -x^2+14x +5$
\item $(2x-5)(3x+2)=6x^2$ \label{en_dvp}

\end{enumerate}
\end{multicols}

\begin{sol}
Ici, on peut se ramener à l'équation $ax^2+bx+x=0$ (et àla méthode du $\Delta$) en transposant dans le  membre de gauche. Pour \ref{en_dvp}, il faut penser à développer le membre de droite.

On peut vérifier ses solutions à la calculatrice, mais il faut savoir détailler les calculs.
\end{sol}


\begin{sol*}
\begin{enumerate}[label=\bf{\alph*})\,]
	\item $x^2+2x=3 \iff  x^2+2x-3 =0$
	
	$\Delta =2^2+4 \cdot 1 +3 =16$. Donc $x_1 = \frac{-2+4}{2}=1$ et $x_2 = \frac{-2-4}{2}=-3$
	
	\item $x^2+40 = -13x \iff x^2+13x+40 = 0$.
	
	$\Delta = 13^2-4 \cdot 1 \cdot 40 = 169 - 160 =9$. Donc $x_1=\frac{-13+3}{2} = \frac{-10}{2}=-5$ et $x_2=\frac{-13-3}{2} = \frac{-16}{2}=-8$.
	
	\item $x^2+5x = -x^2+14x +5 \iff x^2+5x+x^2-14x-5 = 0 \iff 2x^2-9x-5=0$
	
	$\Delta = 9^2+4 \cdot 2 \cdot 5 = 81+40 = 121$ donc $x_1=\frac{9+11}{4} = 5$ et $x_2=\frac{9-11}{4} =-\frac{1}{2}$ 
	
	\item $(2x-5)(3x+2)=6x^2 \iff 6x^2+4x-15x-10 = 6x^2 \iff  -11x-10 =0 \iff x = -\frac{10}{11} \approx -0,909$
	
	
	
	
	
\end{enumerate}	
\end{sol*}	


\end{exo}



\begin{exo}[type=solution]
Avec la calculatrice, mais en détaillant les calculs, résoudre les équations suivantes :

{\it On exprimera les solutions de manière exacte puis arrondies à 0,01 près}


\begin{enumerate}[label=\bf{\alph*})\,]
\item $5x^2-5x+3 = 10x^2+2x-5$
\item $(x+2)(2x-5) = x^2+x +2$
\item $2x^2-5x=\frac{2}{3}$ 
\item $\frac{2}{x-2}=3x-1$ \label{equa_quotient_1}

\end{enumerate}

\begin{sol}
Pour \ref{equa_quotient_1}  on peut utiliser $\frac{a}{b}=c \implies a = b \cdot c$ 
\end{sol}


\end{exo}





\groupexos{Formule du sommet}



%\begin{exom}{\refmethode{methode_sommet}}
%
%	Calculer les coordonnées du sommet de la parabole d'équation $y=x^2-8x+5$.
%\end{exom}


\begin{exo}[title=sans calculatrice]

On a représenté à la calculatrice 4 paraboles dont voici les équations. Indiquer pour chacune d'elle si la forme est convexe ou concave et calculer les coordonnées du sommet.

\begin{multicols}{2}
$y=2x^2-8x+10$

$y=5x^2+12x-3$

$y=-x^2+8x-20$

$y=-x^2+6x+8$
\end{multicols}

 Enfin retrouver le graphique correspondant.


\begin{tikzpicture}[scale=1,general]
    \window{-3}{5}{-20}{20} 
    \begin{windowsratio}
    \draw[xstep=1,ystep=10,grid] (\Xmin,\Ymin) grid (\Xmax,\Ymax);
    \axeH;\axeV;\tickX;\tickY[10];
    \node[below left] at (0,0) {0 };
    \clip (\Xmin,\Ymin) rectangle (\Xmax,\Ymax);   
	 \draw[samples=100,domain=-3:5,courbe] plot(\x,{-(\x)^2+6*\x+8});
	 \draw[samples=100,domain=-3:5,courbe,color=red] plot(\x,{2*(\x)^2-8*\x+10}); 
	  \draw[samples=100,domain=-3:5,courbe,color=black] plot(\x,{5*(\x)^2+12*\x-3}); 
	  \draw[samples=100,domain=-3:5,courbe,color=orange] plot(\x,{-(\x)^2+8*\x-20});
	  \node[below] at (3,15) {\color{blue} $f(x)$};          		
	  \node[] at (-1.6,12) {\color{red} $g(x)$};  
      \node[right] at (0,-5) {\color{black} $h(x)$};  
      \node[right] at (2.2,-10) {\color{orange} $k(x)$};     	    
    \end{windowsratio}
\end{tikzpicture}


\begin{sol*}
On applique la formule du sommet $x_S=-\frac{b}{2a}$.
\begin{itemize}
\item $y=2x^2-8x+10$

$x_S = \frac{8}{4} =2$, il s'agit donc de la fonction $g$.

$g(2)=8-16+10 = 2$, donc $S(2;2)$.

\item $y=5x^2+12x-3$

$x_S = \frac{12}{10} =-1,2$, il s'agit donc de la fonction $h$.

$h(-1,2)=5\cdot 2,44 -12 \cdot 1,2 -3 = 7,2-14,4-3=-10,2$, donc $S(1,2;-10,2)$.

C'est assez difficile à calculer sans calculatrice, sur le dessin, on voit que le minimum est environ $-10$.

\item $y=-x^2+8x-20$

$x_S = \frac{-8}{-2} =4$, il s'agit donc de la fonction $k$.

$k(4)=-16+32-20 = -4$, donc $S(4;-4)$.


\item $y=-x^2+6x+8$

$x_S = \frac{-6}{-2} =3$, il s'agit donc bien de la fonction $f$.

$f(3)=-9+18+8 = 17$, donc $S(3;17)$.


\end{itemize}
\end{sol*}



\end{exo}




\begin{exo}[title=Sans calculatrice]

Soit $f$ la fonction définie par $$f(x) = -x^2+2x+15 \,.$$

\begin{enumerate}
\item La forme de la parabole représentant cette fonction est-elle convexe ou concave ?

\item Calculer les zéros de la fonction $f$.

\item Calculer les coordonnées du point d'intersection de la parabole avec l'axe des ordonnées .

\item Calculer les coordonnées du sommet de la parabole et indiquer l'équation de l'axe de symétrie.

\item Tracer dans un repère la parabole représentant la fonction $f$ en plaçant précisément les points calculés plus haut et en traçant l'axe de symétrie. 
\end{enumerate}
\end{exo}







\groupexos{Intersections d'une parabole et d'une droite}


\begin{exo}
On considère la fonction $f$ définie par  $f(x)=x^2+2x-1$ et la fonction affine $g$ définie par $g(x)=4x+2$.


%
%\begin{image}
%\begin{tikzpicture}[scale=1,general]
%    \window{-5}{5}{-5}{20} 
%    \begin{windowsratio}
%    \draw[xstep=1,ystep=5,grid] (\Xmin,\Ymin) grid (\Xmax,\Ymax);
%    \axeH;\axeV;\tickX;\tickY[5];
%    \node[below left] at (0,0) {0 };
%    \clip (\Xmin,\Ymin) rectangle (\Xmax,\Ymax);   
%	 \draw[samples=100,domain=-5:5,courbe] plot(\x,{(\x)^2+2*\x-1});
%	 \draw[samples=100,domain=-5:5,courbe,color=red] plot(\x,{4*\x+2});      	    
%    \end{windowsratio}
%\end{tikzpicture}
%\end{image}
%
%



\begin{enumerate}
\item Tracer les courbes de ces deux fonctions pour $x$ compris entre $-5$ et $5$. On prendra 1 carreau par unité en abscisses et 1 carreau pour 5 unités en ordonnées.

\item En utilisant la calculatrice, trouver les coordonnées des points d'intersection de la parabole et de la droite.

\item Retrouver les abscisses de ces points d'intersection en résolvant une équation.

 
\end{enumerate}
\end{exo}





\begin{exo}
On considère la parabole d'équation $y=x^2-4x+6$ et la droite d'équation $y=2x-1$.


\begin{enumerate}
\item Tracer la parabole et la droite pour $x$ compris entre $-1$ et 5. On prendre un carreau par unité en abscisses et un carreau pour 2 unités en ordonnées.

\item En utilisant la calculatrice, trouver les coordonnées des points d'intersection de la parabole et de la droite.

\item Retrouver les abscisses de ces points d'intersection en résolvant une équation. On donnera les valeurs exactes puis des valeurs approchées à 0,01 près.
\end{enumerate}

\begin{sol*}
\begin{enumerate}
\item Graphique : \par
\begin{image}
\begin{tikzpicture}[scale=1,general]
    \window{-1}{5}{-2}{12} 
    \begin{windowsratio}
    \draw[xstep=1,ystep=2,grid] (\Xmin,\Ymin) grid (\Xmax,\Ymax);
    \axeH;\axeV;\tickX;\tickY[2];
    \node[below left] at (0,0) {0 };
    \clip (\Xmin,\Ymin) rectangle (\Xmax,\Ymax); 
    \def \f{(\x)^2-4*\x+6};
    \draw[samples=100,domain=-1:5,courbe]
    plot(\x,{\f});     
    \def \f{2*\x-1};
	\draw[samples=100,domain=-1:5,courbe,color=red]
		plot(\x,{\f});     

      
    \end{windowsratio}
\end{tikzpicture}
\end{image}

\item D'après la calculatrice, les points d'intersection sont $(1,6;2,2)$ et $(4,4;7,8)$. Ce sont des valeurs approchées.

\item 
$
\begin{array}{rcl}
x^2-4x+6 & = & 2x-1 \\
x^2-4x+6-2x+1 & = & 0 \\
x^2-6x+7 & = & 0
\end{array} 
$


$\Delta = 36-4 \cdot 1 \cdot 7 = 36-28 = 8$. Donc les deux solutions sont \par
$x_1 = \frac{6+\sqrt{8}}{2} \approx 4,41$ et $x_1 = \frac{6-\sqrt{8}}{2} \approx 1,59$ 

\end{enumerate}
\end{sol*}	

\end{exo}


\begin{exo}[title=Sans calculatrice]

Soit la fonction $f$ définie par $$f(x)=x^2-4x-5\,.$$

\begin{enumerate}
\item Calculer les zéros de la fonction $f$.

\item Calculer les coordonnées du sommet de la parabole qui représente $f$.

\item Soit la fonction affine $g$ définie par $g(x)=-2x+3$. Calculer les coordonnées des points d'intersection de la parabole qui représente $f$  et de la droite qui représente $g$.

\item Représenter sur un graphique la parabole et la droite. On placera avec précision les points calculés plus haut.
\end{enumerate}
\end{exo}













\groupexos{Problèmes du second degré}



\begin{exo}
 La longueur d'un rectangle excède de 5m sa largeur. Son aire est de 300m$^2$. Trouver les dimensions 
de ce rectangle.

\begin{sol*}
Soit $x$ la largeur. Alors 	la longueur est $x+5$.

$
\begin{array}{rcl}
x \cdot (x+5) &  = & 300 \\
x^2 +5x -300 & = & 0 \\
\end{array}
$

$\Delta = 5^2+4\cdot 1 \cdot 300 = 25+1200=1225$. Les solution sont \par 
$x_1 = \frac{-5+35}{2} =15$ et $x_2 = \frac{-5-35}{2}=-20$.

Donc le rectangle mesure 15cm de large et 20cm de long.

\end{sol*}
\end{exo}

\begin{exo}
Le périmètre d'un rectangle est de 36 mètres. Sachant que sa diagonale mesure 15 mètres, calculer la longueur et la largeur du rectangle.

\begin{sol*}
On note $l$ le ur et $L$ la longueur. 

Donc $L+l=18$ (le demi-périmètre) et $L^2+l^2=15^2$.

Ainsi 
$
\begin{array}{rcl}
(18-l)^2+l^2 & = & 15 ^2 \\
324-36l+l^2+l^2-225 & = & 0 \\
2l^2-36l+99 & = & 0
\end{array}
$

$\Delta = 36^2-4 \cdot 2 \cdot 99 = 504$

Donc $l_1 = \frac{36+\sqrt{504}}{4} \approx  14,61$ ou $l_2 = \frac{36-\sqrt{504}}{4} \approx 3,38$.

La largeur mesure environ 3m38 et la longueur 14m61.

\end{sol*}
\end{exo}



\begin{exo}
	Un ouvrier achève de tourner le lot de 90 pièces. Il calcule alors que s'il avait tourné une pièce de plus par heure, il aurait mis une heure de moins pour exécuter son travail. Combien a-t-il tourné de pièces à l'heure ?

\begin{sol*}
On trouve que le travail dure 10 heures et qu'il a tourné 9 pièces à l'heure. Le correcteur attend une équation \ldots
\end{sol*}	

\end{exo}

