\groupexos{Un groupement d'exercices}

 \begin{exo}[type=solution]
 Effectuer les calculs suivants : \par
 $A=2+2$ \par  $B=\frac{1}{3}+5$ \par  
 $C=\frac{2}{3} \div \frac{3}{5}$
 \end{exo}

 \begin{exo}
 Montrer que les nombres de la forme $a+b\sqrt2$, où $a \in \mathbb{Q}$ et $b \in \mathbb{Q}$, forment un corps pour la multiplication et la division usuelle.
 \begin{sol}
  $A=4$, $B=\frac{16}{3}$, $C=\frac{2}{3} \cdot \frac{5}{3} = \frac{10}{9}$.  
 \end{sol}
 \end{exo}
 
  \begin{exo}
   On définit sur $\mathbb{R}^2$ une addition et une multiplication par les formules :
   $(a,b)+(c,d)=(a+c,b+d)$, $(a,b)(c,d) = (ac,ad+bc+bd)$.
   
   \begin{enumerate}
   \item Montrer que $\mathbb{R}^2$ devient ainsi un anneau commutatif à élément unité.
   \item Un élément non nul $a$ d'un anneau commutatif $A$ est dit un \textsl{diviseur de zéro} s'il existe un élément non nul $y$ tel que $ay=0$.
   
   Trouver les diviseurs de zéros de l'anneau considéré.
   \end{enumerate}
   
  \end{exo}