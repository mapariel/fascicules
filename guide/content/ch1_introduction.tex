\chapter{Introduction}

\backgroundimage{img/ch1}
\thispagestyle{chapterpage}

\newpage
\section{Quelle est l'idée ?}

Ce projet a pour but de permettre la création d'un manuel collaboratif de mathématiques. Bien sûr, c'est un peu restrictif, on peut aussi envisager tout autre type de manuel \ldots 

Le manuel sera divisé en chapitres. Chaque chapitre pourra contenir trois parties : une partie de cours, une partie d'exercices (avec corrigés en fin de manuel) et une partie activités.

Le but est de permettre à plusieurs auteurs de collaborer, sans avoir à se préoccuper de la présentation. En respectant quelques contraintes, leur contenu sera harmonieusement intégré à l'ensemble. 

La philosophie derrière tout \c{c}a est assez éloignée du \href{https://fr.wikipedia.org/wiki/What_you_see_is_what_you_get}{WYSIWYG} (What You See Is What You Get). Il s'agirait plutôt de \href{https://fr.wikipedia.org/wiki/What_you_see_is_what_you_mean}{WYSIWYM}  (What You See Is What You Mean). Ou encore de dissocier  le fond  de la forme, le contenu de la présentation. 


\section{Pourquoi \LaTeX ?}

Afin de permettre de dissocier le fond de la forme, et aussi d'obtenir un résultat équivalent aux manuels du commerce, et sans passer par un éditeur, un programme existe depuis les années 80 : \href{https://fr.wikipedia.org/wiki/LaTeX}{\LaTeX}.

Ce programme a l'avantage de permettre d'écrire des équations mathématiques, et de créer des figures. Ce qui est assez intéressant quand on veut faire un manuel de mathématiques.

Enfin, \LaTeX est un programme libre, vous pouvez l'installer gratuitement sur votre ordinateur personnel.

La communauté des utilisateurs de \LaTeX est assez riche, et vous trouverez sur Internet une foule d'information. La difficulté étant,  pour le débutant, d'y faire le tri.

Voici quelques liens, parmi les premiers qu'on trouve en faisant une recherche sur le web :

\begin{itemize}
\item Le site des \href{http://www.tuteurs.ens.fr/logiciels/latex/}{tuteurs} de l'ENS. 

\item Le \href{https://www.ljll.math.upmc.fr/privat/documents/manuelLatex.pdf}{petit guide} pour les débutants en \LaTeX, que l'on trouve sur le site du laboratoire Jacques-Louis Lyons.
\end{itemize} 

J'ai appris \LaTeX~à l'aide du livre de Christian Rolland, \LaTeX par la pratique (Edition O'ReillY). Excellent ouvrage.

Il existe plusieurs distributions pour installer \LaTeX, une des plus populaires se nomme MiKTex (\url{https://miktex.org/}). Ensuite pour éditer ses textes, il y a là encore l'embarras du choix. J'utilise TeXstudio (\url{https://www.texstudio.org/}). Ces deux liens devraient vous permettre de démarrer, que vous soyez un utilisateur de Windows, MacOS ou de Linux.

\section{Pourquoi ce guide ?}

L'objectif de ce guide est de permettre à tout professeur de mathématiques qui le souhaiterait, de participer à la rédaction de manuels de mathématiques collaboratifs. Évidemment, l'apprentissage de \LaTeX peut sembler à première vue assez délicat, surtout pour quelqu'un qui n'est pas familier avec le code informatique.  



%\begin{lesson}

%\end{}